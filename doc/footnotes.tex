\section{Footnotes}
\begin{thebibliography}{99}

\bibitem{manin1997} Refer to \cite{manin1997}.

\bibitem{social-choice} Social choice started as non rooted in a rational model of humans: the first theorems (Condorcet) modeled humans as realization of some random variables and did not come with micro-foundations for these random processes. The 1950's social choice theory became more micro-founded/rational.

\bibitem{values-prefs} Or, more precisely, ideas like values, beliefs, and norms have just been folded into preferences. I don't mean to minimize the amount of work done on these topics: Outside the SIDT, these intersubjective concepts have been modeled in fields like sociology. And there is hugely successful work extending economics or game theory in these directions, but it remains peripheral. For instance, in microeconomics and game theory there are well-known theories of norm emergence; there are alternatives to rational choice theory like Ruth Chang's or Isaac Levi's. There are also larger efforts, like institutional and organizational economics. But (with the exception of transaction cost econ) these never make it into the SIDT that is used to design or justify institutions. Other innovations which do make it in (like behavioral economics or social network analysis) have remained context-free.

\bibitem{revealed-pref} When social theories equate revealed preference with benefit, anyone who can manipulate another's choice (which is supposed to reveal a certain preference) can count as benefiting them.

\bibitem{klingefjord} See \cite{klingefjord2024}

\bibitem{xuan} See Xuan...

\bibitem{sen} Amartya Sen

\bibitem{people-prefs} \textit{People's preferences change productively over time as they better understand what's important to them, face new contexts, and realize aspects of situations they didn't previously consider. That means that some people have better preferences in some contexts. But mechanisms based on these prevailing social theories can't detect them. These mechanisms miss the opportunity to harness moral learning, to let us reach collectively towards something that's not just the average.}

\bibitem{conitzer} Conitzer moral mechanism

\bibitem{klingefjord2} See \cite{klingefjord2024} again

\bibitem{weyl} Glen Weyl put it well:
``[Currently models feature] a binary between Individuals, conceptualized as largely pre-social, independent ultimate loci of value/preference/good/belief, and some global coordination device variously referred to as the social planner, the mechanism designer, the impartial observer, God or, most commonly, The State.

It is from this binary that the initially paradoxical combination of extreme individualism and authoritarian technocracy that characterizes [this view] arises.''

\bibitem{edelman} A longer treatment of these ideas is in ``Nothing to Be Done'' by Joe Edelman. Look for the paragraph that starts:
``Each of these fields is subject to another kind of test, besides its ability to predict human choices. Each field has a model of human beings which focuses on certain skills or abilities or activities of the human subject.''

\bibitem{terms} Some of these terms allow people to advocate for a policy direction, some allow them to find each other, some allow them to advocate for leaders that serve their previously-unnameable interests. `Modern Social Imaginaries /`

\bibitem{voting} There's even a tendency to say something like ``I guess if people are voting for social isolation, with their dollars and at the ballot, then a good society is one of social isolation! (So long as it's free and fair.)''

\bibitem{taylor} [Taylor, ``Modern Social Imaginaries''] Taylor argues that historically, this co-option ends up undermining the power of the incumbent system.

\bibitem{effects} These effects are considered fairly well-supported in political theory and political science. [Weber, Taylor, some recent political scientists]

\end{thebibliography}

\begin{small}
\textbf{Rhetoric / Use Flywheel:}

Let us suppose this pans out. This will then change people's ideas of what a good society looks like. Currently, the dominant social ideals are freedom and fairness. It is not so surprising that—in a society dominated by markets and voting, where we think of ourselves as entrepreneurs and citizens—our ideal society is one of freedom and fairness. Other notions—like having meaningful lives, advancing science and art, having strong communities—these seem attractive, but they don't fit as cleanly into our markets/voting-dominated vocabulary as freedom and fairness do.

But note that the this depends on the institutional mix. Other mechanisms lead to other ideals. For instance, the spread of wikis and open source has popularized the idea that a good society should be open to 'pull requests' of some form, leading to 'read/write society' wiki-based governance ideas in Taiwan and other places. One can imagine that Twitter's Community Notes might popularize a notion that a good society is 'algorithmically sober'.

In general, whenever a new institution seems better along a \textit{new dimension} (it is not just more free or more fair), a new rhetoric emerges. It's unlikely that 'read/write society' or 'algorithmically sober' will overtake freedom and fairness as the qualities most heavily advocated for in social arrangements, but we think that some other term, not coined yet, probably will. This is how political paradigm shifts have always gone.

We don't know which new terms will take hold—they might be about ``surfacing collective wisdom,'' ``strengthening social fabric,'' or ``enabling diverse forms of life''—but they will become frameworks for reimagining social possibilities.

Should such a positive feeling and new rhetoric emerge around our flagship deployments, we will encourage it. We will help encourage participants to articulate the advantages of these new systems, and share what it's like to participate in them.

Terminology can be a decisive factor in which groups manage to coordinate politically. For instance, it's hard for the working class to find each other without the term 'working class'; it's hard to fight for collective ownership without ideas like 'worker coop'; it's hard to fight for local self-determination without concepts like 'freedom', 'subsidiarity', or 'laissez-faire'.

Many communities would ideally band together in common interest, but for now they can't even find one another, or have nothing clear to advocate for. There are no terms yet.

Once there are new terms—once people begin to see themselves as a ``values-driven'' (rather than just as entrepreneurs, informed citizens, Democrats, or Republicans) and band together with other values-driven people to further values-driven forms of social organization—it will be hard to reverse.

Eventually, the legitimacy of political arrangements will be defined by a vocabulary which today isn't even in play.
\end{small}