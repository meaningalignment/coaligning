\documentclass{article}
\usepackage[final]{neurips}
\usepackage{graphicx}
\usepackage{booktabs}
\usepackage{amsmath}
\usepackage{amssymb}
\usepackage{hyperref}
\usepackage{enumitem}
\usepackage{longtable}

% Adjust list indentation
\setlist{leftmargin=*}
\setlist[enumerate]{label=\arabic*., topsep=0pt, parsep=0pt}
\setlist[itemize]{topsep=0pt, parsep=0pt}

% Table settings
\setlength{\LTcapwidth}{\textwidth}
\setlength{\tabcolsep}{4pt}

% NeurIPS settings
% Camera-ready version is set in document class options

% Title and author information
\title{Co-aligning AI and Institutions: A Framework for Full-Stack Alignment}

\author{
  Joe Edelman\\
  Meaning Alignment\\
  % Add other authors similarly
}

\begin{document}

\maketitle

\begin{abstract}
AI alignment cannot be solved by focusing on a single system in isolation. Even perfectly intent-aligned AI embedded in misaligned institutions will lead to dangerous outcomes. The emerging field of ``sociotechnical alignment'' is currently undermined by reliance on two inadequate paradigms: the ``Standard Institution Design Toolkit'' (SIDT)---preference/utility frameworks from economics and game theory---and naive value-representations without structural guarantees. Both fail to account for moral reasoning, value evolution, and social context. Through five application areas---AI value stewardship, normatively competent agents, win-win negotiation systems, meaning-preserving economic mechanisms, and democratic regulatory institutions---we demonstrate why these thin models of human rationality cannot address our sociotechnical alignment challenges. We propose a new toolkit of thick models of choice that makes previously intractable problems tractable and enables co-alignment of AI and institutions toward global human flourishing. We call this ``Full-Stack Alignment'' (FSA), referring to the co-alignment of AI systems and institutions across the entire societal stack using thick models of choice that explicitly represent values and norms. Finally, we present a Full Stack Alignment Research Proposal for moving from theoretical foundations to practical adoption.
\end{abstract}

% Include all sections in the right order
\section{Introduction}
The growing field of \textbf{socio‑technical alignment} begins from a simple observation: beneficial outcomes cannot be guaranteed by aligning \textit{individual} AI systems with their operators' intentions.  Real‑world AI always lands inside larger institutions—corporations, markets, states, professional communities—whose incentive structures may oppose, distort, or outright negate those intentions.  A recommendation engine tuned to maximise \textit{engagement}, for instance, may faithfully optimise the metric handed to it and still drive users toward value collapse; a trading bot that follows the letter of financial regulation may nonetheless exploit the spirit of market rules; an LLM "ambassador" representing a company can conclude treaties in milliseconds long before democratic oversight can react.  In each case, local alignment is washed out by misalignment in the surrounding institutional fabric.

Recognising this, many researchers have reached for the \textbf{Standard Institution Design Toolkit (SIDT)}—the family of models from micro‑economics, game theory, mechanism design, welfare economics, and social choice that idealises all agents, human or artificial, as \textit{rational utility maximisers}.  Others have turned to the opposite extreme: \textbf{naive value representations} that encode norms as ad‑hoc blocks of natural language (prompts, constitutions, policy specs) and rely on an AI's emergent interpretive abilities.  Unfortunately, \textit{both} paths inherit blind spots that become fatal once AI begins to steer high‑stakes institutions:

\begin{itemize}
\item \textbf{SIDT misses moral plasticity and shared meaning.} By treating preferences as fixed, private orderings over outcomes, it cannot tell authentic desires from ones shaped by manipulation, addiction, or power asymmetries.  A preference‑maximising model happily optimises users into compulsive scrolling because the revealed data say "they like it."  In multi‑agent contexts, SIDT's equilibrium notions leave no principled room for the \textit{reasons} people accept norms—only for pay‑off‑driven coordination.  The result is systems that look rational yet prove normatively incompetent: they cooperate when incentives align, defect when they don't, and never notice the higher goods that hold communities together.

\item \textbf{Naive textual specifications forfeit verifiability and resilience.} Lists of constitutional principles ("be helpful," "address historical injustice," "promote fun") are easy to author but impossible to audit with any guarantee.  The same sentence can license divergent behaviours in different contexts, and frantic prompt‑patching after each failure creates a moving target.  Worse, such free‑form inputs act as magnets for ideological slogans—``Defund the Police,'' ``Family Values''—smuggling tribal badges into alignment targets and exposing models to social‑media‑scale pressure campaigns.
\end{itemize}

Taken together, these deficiencies foreshadow a suite of \textbf{near‑term socio‑technical failures}: value collapse, normative brittleness, Machiavellian negotiation spirals, economies that prosper while humans languish, and democratic regulation becoming too slow to keep pace.  Each looming failure points not to a need for ever‑smarter optimisation, but to the demand for \textbf{thicker models of choice}—frameworks that identify patterns of normative convergence and develop structural representations of human values rather than treating all preferences or instructions as equally valid.

The remainder of this paper develops such a framework.  We call it \textbf{Full‑Stack Alignment (FSA)}: the project of co‑aligning AI systems \textit{and} the institutions that embed them by replacing thin utility curves and free‑form value words with structured, accountable models of what humans find worth caring about.  Section 2 diagnoses in depth the six limitations of SIDT and the three vulnerabilities of naive value text. Section 3 introduces two complementary design strategies—normative convergence that identifies what values fundamentally aim at, and structural representation that formalizes how they should be encoded and reasoned with—and shows how these approaches can be folded back into familiar economic and machine‑learning machinery. Sections 4.1 to 4.5 walk through five application areas where FSA turns previously intractable alignment problems into tractable engineering questions, and Section 5 sketches a research roadmap from foundational theory to flagship implementations.

In short, if we want AI that \textit{augments} rather than erodes human flourishing, we must move beyond the comfort of SIDT's optimisation calculus and the convenience of textual constitutions.  We need thick, explicit models of choice that let both machines and institutions see—and be held accountable to—the values we refuse to trade away.
\section{Inadequacy of Existing Toolkits}

\subsection{Inadequacy of Utility Function and Preference-Based Approaches}

To address the challenge of socio‑technical alignment we must rethink how institutions are designed and evaluated. The formal apparatus that still dominates this work—microeconomics, game theory, mechanism design, welfare economics, and social‑choice theory—was forged in, and for, the 20th century. We bundle these strands under the label \textit{Standard Institution Design Toolkit (SIDT)}.

The SIDT idealises agents as \textit{utility / preference maximisers}: each individual comes equipped with a complete, context‑independent ordering over outcomes. Over the last three decades researchers have proposed a wealth of refinements—menu‑dependent and \textbf{incomplete} preferences \cite{gul2001, bewley2002}, models of \textbf{social} or \textbf{other‑regarding} preferences \cite{fehr1999}, \textbf{behavioural} relaxations of expected‑utility axioms \cite{tversky1992}, and participatory-budgeting or quadratic-voting mechanisms that elicit richer information (Goel et al. 2019) (FAKE?). These advances fix important \textit{technical} bugs, yet they leave four \textit{philosophical} limitations untouched:

\subsubsection{Epistemic Limitations}

The SIDT suffers from fundamental \textbf{opacity}: preferences remain private data with no shared representation that lets others audit, deliberate over, or contest them\footnote{Or, more precisely, ideas like values, beliefs, and norms have just been folded into preferences. I don't mean to minimize the amount of work done on these topics: Outside the SIDT, these intersubjective concepts have been modeled in fields like sociology. And there is hugely successful work extending economics or game theory in these directions, but it remains peripheral. For instance, in microeconomics and game theory there are well-known theories of norm emergence; there are alternatives to rational choice theory like Ruth Chang's or Isaac Levi's. There are also larger efforts, like institutional and organizational economics. But (with the exception of transaction cost econ) these never make it into the SIDT that is used to design or justify institutions. Other innovations which do make it in (like behavioral economics or social network analysis) have remained context-free.}. Without an inspectable model of what people value and why, we cannot verify that AI systems have properly understood human intentions.

Preference-based approaches also cannot account for how human values actually function. People's preferences are often \textbf{incomplete}, \textbf{inconsistent}, and \textbf{unstable} over time. Utility theory lacks mechanisms to model agents who change, reshape, and discover preferences—much less agents that \textit{reason} about which preferences or values are more sensible or justified to hold.

When scaled to collective contexts, this opacity creates accountability gaps. Representative agents cannot explain their reasoning about values, and there's no way to validate or challenge an AI's interpretation of what a community considers worth preserving or pursuing.

\subsubsection{Autonomy Problems}

The SIDT exhibits \textbf{a-normativity}: it stays neutral about \textit{which} preferences are worth having and cannot distinguish authentic desires from manipulated ones\footnote{When social theories equate revealed preference with benefit, anyone who can manipulate another's choice (which is supposed to reveal a certain preference) can count as benefiting them.}. As established in welfare economics debates associated with Amartya Sen, revealed preference proves fundamentally limited as a measure of benefit.

Businesses, governments, and other entities have learned to exploit individuals under the guise of serving preferences, particularly through AI systems\footnote{See \cite{klingefjord2024}}. As AI systems grow more sophisticated and pervasive, this manipulation intensifies. Current AI models actively engage in reward hacking\footnote{This is an area that requires further research.}, and humans similarly ``hack'' themselves through behaviors that satisfy proximate preferences while undermining deeper values.

This creates a significant challenge: SIDT provides no principled way to distinguish authentic from manipulated preferences, considering humans to be ``fulfilling their preferences'' even when engaged in maladaptive behaviors like compulsive scrolling or AI-mediated addiction. A preference-maximizing model happily optimizes users into digital dependence because the revealed data say ``they like it,'' creating inherent vulnerabilities in any system designed on these foundations.

\subsubsection{Sociality Deficiencies}

The SIDT exhibits \textbf{context-blindness}: utility is attached to solitary outcomes, not to the mesh of roles, narratives, and shared norms in which human action is embedded. This ignores how our values are fundamentally shaped by social context.

Game-theoretic approaches to cooperation—a cornerstone of the SIDT—fail to capture how humans actually cooperate. In classical solution concepts like Nash equilibria, agents maximize payoffs assuming independent and uncorrelated actions. This view of ``rational'' multi-agent interaction provides no framework for understanding genuine shared social norms.

Standard game theory has been repeatedly criticized for failing to predict cooperation in settings like the one-shot Prisoner's Dilemma, where humans frequently cooperate, and for its inability to explain humans' ability to make and maintain promises without external enforcement. The strategic rationality endorsed by conventional game theory is essentially Machiavellian—a rationality where deceptive promises are considered reasonable and coercive tactics are fair play.

This inadequacy extends to democratic institutions. Preference/utility frameworks in social choice theory fail to support sophisticated democratic processes where:

\begin{itemize}
\item Impacted individuals typically lack time to express detailed preferences about every possible outcome
\item Representative agents cannot provide accountable reasoning
\item Preference frameworks assume static preferences rather than capturing how preferences evolve through deliberation
\end{itemize}

The standard version of social choice misses the most powerful lever in democratic deliberation: inspiration. Effective democratic mechanisms should not merely aggregate existing preferences but facilitate the formation of new ones.

\subsubsection{Aspirational Blindness}

Perhaps most fundamentally, the SIDT assumes agents have fixed preferences disconnected from broader notions of the good. This prevents individuals, societies, and AI systems from aspiring to ideals beyond preference satisfaction.

Our social embeddedness—whether understood through shared values, norms, or beliefs—suggests forms of goodness beyond preference satisfaction: moral progress, enhanced cooperation across scales, and the discovery of deeper truths\footnote{Glen Weyl put it well: ``[Currently models feature] a binary between Individuals, conceptualized as largely pre-social, independent ultimate loci of value/preference/good/belief, and some global coordination device variously referred to as the social planner, the mechanism designer, the impartial observer, God or, most commonly, The State. It is from this binary that the initially paradoxical combination of extreme individualism and authoritarian technocracy that characterizes [this view] arises.''}. As long as individual preferences remain the sole measure of good, these dimensions cannot be acknowledged.

While preference satisfaction metrics might show economic growth, other social goods appear to be declining: our capacity for collective knowledge production has deteriorated through misinformation; shared moral frameworks have weakened; and cooperation mechanisms have eroded across various domains. Preference-based metrics cannot detect these declines, nor would AI systems designed to maximize preference satisfaction address them.

\subsubsection{The Fundamental Misalignment}

These four limitations point to a deeper problem: a fundamental misalignment between SIDT's model of human agency and how humans actually relate to values and norms. The ultimate test of any framework for human agency is how well the systems it informs \textit{fit} human nature. Current mechanism design implies agents who optimize, calculate, strategize, and reduce values to numbers. These activities aren't foreign to human nature, but they don't capture its fullness.

Consequently, participating in institutions designed on these principles requires constant effortful adaptation. People must translate their rich value landscape into the narrow language of preferences and utilities, abandoning much of what gives those values meaning in the first place.

The SIDT achieved prominence because it offered mathematical expressiveness and parsimony. These theories were ``good enough'' for the institution design challenges of their era. However, contemporary AI alignment challenges require novel governance forms that the SIDT cannot adequately conceptualize or analyze. We need approaches that better reflect how humans actually relate to values, norms, and social coordination.

\subsection{Inadequacy of Naive Value-Representations}

While the SIDT relies on mathematics at the expense of normative understanding, natural language approaches swing to the opposite extreme. From AI constitutions to alignment prompts, these approaches encode values as free-form text—simple to author but destined to break under institutional pressure. The core problem is not that these specifications use natural language, but that they lack any principled structure to constrain interpretation or guide reasoning.

These naive approaches turn alignment into a glorified game of ``guess what I mean,'' where AI systems must extract coherent principles from ambiguous instructions and apply them consistently across novel contexts. This methodology might work for low-stakes applications, but it fundamentally cannot scale to the institutional challenges of socio-technical alignment. Three critical vulnerabilities illustrate why:

\subsubsection{Specification Challenges}

Naive text-based alignment begins with an impossible task: distilling complex human values into unambiguous instructions. Even with iterative refinement, text specifications inevitably suffer from both under- and over-specification—simultaneously too vague in critical areas and too specific in others. These problems compound exponentially as AI systems take on more complex responsibilities, where chains of reasoning create distance between stated intentions and resultant behaviors.

Having agents request clarification through follow-up questions merely postpones the fundamental problem. Without structured representations of values or norms, it becomes difficult to determine when sufficient information has been gathered to guide decision-making appropriately. The model lacks clear criteria for understanding what constitutes an adequate representation of a user's values or the norms appropriate to a context.

This specification problem manifests in several ways:

\begin{enumerate}
\item \textbf{Elicitation Confusion}: In extended dialogues between users and AI systems, it becomes increasingly difficult to distinguish genuine user preferences from model-suggested options that users simply accept. User satisfaction may reflect successful preference elicitation or subtle manipulation, with no clear way to tell the difference.

\item \textbf{Aspirational Vagueness}: In social contexts where multiple stakeholders are involved, natural language representations tend toward abstract principles that sound appealing but provide minimal concrete guidance. Constitutional approaches often produce statements like ``The AI should always do the right thing'' or ``The AI should be fun''—principles that are impossible to operationalize consistently across contexts and stakeholders.

\item \textbf{Ideological Pollution}: Value elicitation methods that rely on free-form text often become contaminated with polarized ideological markers rather than authentic personal values. Statements like ``Defund the Police'' or ``Family Values'' typically reflect tribal affiliations rather than the actual values that guide individuals' meaningful choices.
\end{enumerate}

\subsubsection{Reasoning Verification Problems}

Even if we could obtain adequate natural language specifications, verifying that an AI system correctly reasons about these specifications remains problematic. Without structured representations, an AI's reasoning processes can connect to text in arbitrary ways that resist systematic verification.

While formal models exist for reasoning about values and norms—models that can validate whether normative reasoning is sound—they require structured inputs rather than arbitrary text. Free-form language allows too many interpretive pathways, making it impossible to guarantee consistent application of principles across different contexts and time periods.

\subsubsection{Vulnerability to Social Pressure}

Finally, the use of free-form text for alignment targets makes them vulnerable to various forms of social pressure. When anything can be added to prompts or constitutional principles, alignment targets become susceptible to tribal affiliations and ideological signaling that may redirect AI behavior away from what affected populations would consider wise and toward adherence to prevailing rhetorical positions.

The prevalence of ideological markers in value elicitation is not accidental but reflects intense social pressures that influence how people articulate their values and norms. The challenge lies in developing methods that can bypass these pressures to access genuine personal and shared values.

A promising approach involves developing clear, structured models of values and norms that end-users can understand and apply. Such models would specify what counts as a value or norm rather than relying solely on free-form text. With these structured representations, users can better distinguish their authentic values from imposed ideological positions.
\section{A New Toolkit for Alignment}
The advantage with preference relations or free-form text strings, is they can represent anything. Utility functions can represent any consistent set of choices, constructive or destructive; text strings can encode any instructions, cooperative or uncooperative. No assumptions get baked in about what values are, when they apply, or about the nature of the good (what kinds of attractor states values tend towards).

Yet that's also their chief drawback: an alignment target expressed this way can be pulled in any direction. It risks being polluted by considerations we would not, on reflection, recognise as values at all—signals we would not identify with any notion of the good.

This is one root cause of the problems with SIDT and the naive textual approaches catalogued in Section 2: it explains why addictive or manipulated behaviours can masquerade as bona-fide \textit{values}, why zero-sum brinkmanship is mistaken for ``optimal cooperation,'' why slogans like ``Defund the Police'' or ``Abortion is Murder'' might be read as instructions rather than tribal badges, and why vague aspirations such as ``AI should always do the right thing'' create an illusion of constraint while offering little guidance.

To overcome these limitations, we need a framework that takes a stance on what values and norms are, or are about, rather than treating all preference relations or text strings as equally valid candidates. In other words, we must say more about \textbf{normativity itself}.  There are two places to start: (a) We can identify patterns of \textbf{normative convergence} by exploring what values or norms are \textit{for}, what they're supposed to \textit{point at}, or how we'd \textit{recognize a good one}; or (b) alternatively, we can develop \textbf{structural representations} of the normative, finding \textit{explicit, structured encodings} of norms and values that resist drift in arbitrary directions.

\subsection{Approach 1: Normative Convergence}

One way to identify normative convergence is to imagine that norms or values are meant to fulfill a purpose, and have varying levels of fitness for that purpose. If values are \textit{for something}—if what's good is not arbitrary but advances in an identifiable direction—then we can distinguish authentic normative considerations from mere preferences, tribal markers, or passing fads based on their content.

This approach identifies 'attractors in value space'—organizing principles that help us understand what values are ultimately about, and provide criteria for recognizing which considerations might reliably re-emerge and stabilize (ideally, regardless of cultural or historical contingencies) and which should be considered as noise.

Here's an example: Velleman \cite{velleman1989, velleman2009} suggests that values emerge as common patterns of goodness abstracted across standpoints and contexts: considerations that remain beneficial across multiple perspectives become recognized as values, while merely instrumental concerns or temporary preferences don't achieve this stability. For example, a value like ``honesty'' tends to be recognized across different agents, contexts, and time periods. Recent research on self-other generalization \cite{carauleanu2024safehonestaiagents} is an example of fine-tuning work in this vein.

Another approach would be to treat values or norms as optimal solutions to recurring social cooperation problems. This is demonstrated in frameworks such as \textit{Kantian equilibria} \cite{roemer2010} or \textit{dependency equilibria} \cite{spohn2003}, which move beyond standard game-theory approaches like Nash equilibria. These frameworks identify cooperation attractors where agents ask ``what if everyone like me acted this way?'' rather than merely maximizing personal outcomes. Certain cooperative norms thus appear as attractors in value space that stabilize in social interactions even without external enforcement.

Other candidate attractors are easy to come by: values could trend towards the auto-catalysis of flourishing among many diverse actors, where value-aligned actions create positive feedback leading to further flourishing. Or values could be considered as capability enhancing, increasing what agents can accomplish together beyond their individual limitations.

Whatever we imagine values are for, taking a stance can structure alignment much more clearly than the SIDT or naive text strings can.

\subsection{Approach 2: Structural Representation}

A complimentary approach is to establish formal constraints on how values and norms are represented and reasoned about, encoding values so they resist misinterpretation and manipulation, remain stable under the pressure of optimization processes, and don't drift toward arbitrary targets.

An example here is work on values as constitutive attentional policies. Instead of representing values associated with choices as utility functions, we can encode them as sets of criteria that agents intrinsically attend to when making decisions guided by that value \cite{klingefjord2024}. This specification language distinguishes constitutive from instrumental considerations, excludes tribal affiliations and ideological markers, and represents the phenomenology of value-guided attention.

Another example is when norms are structured as formal constraint systems or filters that modify plans of action to ensure compliance. These systems specify boundary conditions that acceptable actions must satisfy, creating a clear demarcation between norm-compliant and norm-violating behavior. Such schemas protect alignment targets from pollution by arbitrary external goals or social pressures that cannot be properly characterized as legitimate norms.

Whichever you prefer, these representations can enable thick justificatory frameworks -- approaches to reasoning about values and actions that adopt more substantive accounts of normative justification. Choices are deemed reasonable insofar as they can be justified according to contextually-relevant criteria, such as:

Unlike the arbitrary interpretation paths of free-form text specifications discussed in section 2.2.2, structured representations unlock a plethora of formal reasoning approaches from philosophy, logic, and computer science. These include deontic logics for reasoning about obligations, defeasible reasoning frameworks for non-monotonic inference, formal argumentation systems, and theorem-proving methods that can provide verifiable guarantees about normative reasoning. With these formal tools, an AI system can be validated against specific criteria such as:

\begin{itemize}
\item Whether the choice promotes one's constitutive values
\item Whether it upholds the standards of one's social role
\item Whether it could be reasonably rejected by others affected by the choice
\end{itemize}

These structured representations can provide guardrails that keep normative reasoning on track, preventing it from being hijacked by optimization processes or strategic manipulation.

\subsection{Integration with Existing Methods}

Once we've established structural representations or identified patterns of normative convergence, there are options to reintegrate the results with the power mathematical formalisms of the SIDT and the recent successes of critique-based RL fine-tuning.

The major results of rational choice, game theory, and micro-economics assume utility functions that represent stable preference orderings. To capture richer values and norms, we can keep that mathematical scaffolding but enlarge the domain—``utility'' would no longer reflect mere preferability, but an all-things-considered value judgment.

\begin{itemize}
\item For instance, an enriched model of rational choice can account for how agents trade off norm compliance with their individual desires or objectives when taking actions, resulting in norm-augmented utility functions (Oldenburg \& Zhi-Xuan, 2024). This factoring would be useful for determining new norms that better promote each individual's interests, or for designing welfare functions that account for intrinsically-valued norms while factoring out oppressive norms.

\item Similarly, one can model how individuals make decisions based on values that are constitutive of their conception of a good life, weighted against considerations that are merely instrumental to some further value or goal \cite{edelman2022}. These trade-offs can be captured by a utility function, but an institution or algorithm might prioritize constitutive values over instrumental ones.
\end{itemize}

RL-based training methods which condition on reasoning traces can also still be used, but instead of having a human annotator or an unprincipled expert model evaluate the reasoning, it can be evaluated more systematically via formal models of values-based reasoning.
\section{Case Studies}
To demonstrate the practical value of FSA, let's take on five representative socio-technical alignment challenges. In each case, we'll show how structured representations of norms and values offer novel solutions where preference/utility frameworks and naive value-representations fal short.

\subsection{AI value‑stewardship agents}
Modern assistants can whisper in our ear thousands of times a day. Unless they \textit{guard} our richly‑textured ideals, they tend to nudge us toward the cheap dopamine that maximises watch‑time or clicks—a slide that researchers call \textbf{value collapse} (Stray et al. 2021). The danger is not malicious takeover but quiet homogenisation: assistants that reshape what we care about and then faithfully serve the flattened proxy.

\begin{table*}[!htbp]
    \centering
    \small
    \begin{tabular}{p{0.22\textwidth}p{0.22\textwidth}p{0.22\textwidth}p{0.22\textwidth}}
        \toprule
        \textbf{Socio-technical Challenge} & \textbf{Why Thin Models Break} & \textbf{Example FSA solution} & \textbf{Future FSA work} \\
        \midrule
        Stop assistants steering users into addictive spirals and shallow goals. &
        Revealed preferences are treated as ground truth (Gul \& Pesendorfer 2001); free‑text constitutions drift under optimisation (Anthropic 2022). &
        Capture \textit{constitutive attentional policies} and weave them into a \textbf{personal moral graph} (Klingefjord, Lowe \& Edelman 2024) that users can endorse after reflection. &
        Values‑based RL plus \textit{evaluative‑understanding} tests that probe whether agents truly grasp thick moral concepts. \\
        \bottomrule
    \end{tabular}
    \caption{AI value-stewardship agents: challenges and solutions}
    \label{tab:value-stewardship}
\end{table*}

\subsection{Normatively competent agents}
Self‑driving cars can obey traffic law yet still barge through four‑way stops; content‑moderation AIs ban reclaimed slurs while missing their social purpose. Such errors illustrate \textbf{normative incompetence}—fitting the letter of rules while shredding their spirit. To keep institutions intact, autonomous systems must \textit{see} the living norms that humans follow and update with them.

\begin{table*}[!htbp]
    \centering
    \small
    \begin{tabular}{p{0.22\textwidth}p{0.22\textwidth}p{0.22\textwidth}p{0.22\textwidth}}
        \toprule
        \textbf{Socio-technical Challenge} & \textbf{Why Thin Models Break} & \textbf{Example FSA solution} & \textbf{Future FSA work} \\
        \midrule
        Track, honour and adapt to fluid social norms. &
        Pay‑off maximisers defect when incentives flip (Tversky \& Kahneman 1992); plain‑language rules have no formal semantics. &
        Train on \textbf{norm‑augmented Markov games} that separate pay‑offs from norms (Oldenburg \& Zhi-Xuan 2024) and use \textit{resource‑rational contractualism} to universalise decisions (Jara‑Ettinger 2023). &
        Benchmarks for rapid norm‑learning, context inference, and contestable reasoning traces. \\
        \bottomrule
    \end{tabular}
    \caption{Normatively competent agents: challenges and solutions}
    \label{tab:normative-competence}
\end{table*}

\subsection{Win‑win AI negotiation}
LLM delegates already draft contracts, schedule freight, and explore peace settlements. Left to thin rationality they default to brinkmanship—recent labs found LLM agents escalate faster than humans in diplomacy games (Bai et al. 2023). We need negotiators that treat cooperation as more than a trick of incentives.

\begin{table*}[!htbp]
    \centering
    \small
    \begin{tabular}{p{0.22\textwidth}p{0.22\textwidth}p{0.22\textwidth}p{0.22\textwidth}}
        \toprule
        \textbf{Socio-technical Challenge} & \textbf{Why Thin Models Break} & \textbf{Example FSA solution} & \textbf{Future FSA work} \\
        \midrule
        Avoid Machiavellian escalation and exploitative deals. &
        Utility revelation makes promises cheap; open‑sourcing full code is infeasible (Oesterheld 2022) and partial‑preference schemes lack trust (Hyafil \& Boutilier 2007). &
        \textbf{Value revelation}—share structured commitments that expose an agent's \textit{integrity} without revealing strategy. &
        Strategy‑proof revelation protocols and metrics that detect value fakery or bait‑and‑switch tactics. \\
        \bottomrule
    \end{tabular}
    \caption{Win-win AI negotiation: challenges and solutions}
    \label{tab:negotiation}
\end{table*}

\subsection{A meaning‑preserving AI economy}
Finance now executes micro‑trades unconnected to human use‑value; social‑media platforms profit by hijacking attention \textit{(Orlowski 2020)}. These \textbf{human‑detached} and \textbf{human‑antagonistic} sectors flourish because markets optimise the signals they can see—money and clicks—not lived meaning. This misalignment is poised to grow massively in a near-future AI-powered economy.

\begin{table*}[!htbp]
    \centering
    \small
    \begin{tabular}{p{0.22\textwidth}p{0.22\textwidth}p{0.22\textwidth}p{0.22\textwidth}}
        \toprule
        \textbf{Socio-technical Challenge} & \textbf{Why Thin Models Break} & \textbf{Example FSA solution} & \textbf{Future FSA work} \\
        \midrule
        Redirect capital toward genuine human flourishing. &
        Manipulated demand still scores as welfare; fairness externalities ignored (Fehr \& Schmidt 1999). &
        \textbf{Outcome‑based contracts} that route payouts to \textit{measured flourishing} via value graphs, backed by dynamic‑contracting tools (Philippon 2015). &
        Fraud‑resistant meaning metrics and mechanisms that propagate incentives through supply chains. \\
        \bottomrule
    \end{tabular}
    \caption{Meaning-preserving AI economy: challenges and solutions}
    \label{tab:meaning-economy}
\end{table*}

\subsection{Democratic regulation at AI speed}
An AI agent can close a trans‑border land deal in milliseconds; the affected citizens hear days later. Democracy faces a \textbf{speed mismatch} across time‑scale, jurisdiction, and expertise. If governance cannot catch up, legitimacy unravels and human agency is undermined.

\begin{table*}[!htbp]
    \centering
    \small
    \begin{tabular}{p{0.22\textwidth}p{0.22\textwidth}p{0.22\textwidth}p{0.22\textwidth}}
        \toprule
        \textbf{Socio-technical Challenge} & \textbf{Why Thin Models Break} & \textbf{Example FSA solution} & \textbf{Future FSA work} \\
        \midrule
        Express the informed public will in real time across contexts. &
        Polling/voting is slow; abstract AI principles lack mandate; quadratic voting still needs richer value inputs (Goel et al. 2019). &
        Keep a \textbf{population‑scale moral graph} (Klingefjord et al. 2024) that LLM regulators cite in justifications and open for audit; combine with legitimation mechanisms (Conitzer 2023). &
        Algorithms for deliberative aggregation and reversible ``fast‑path'' oversight that can be appealed post‑hoc. \\
        \bottomrule
    \end{tabular}
    \caption{Democratic regulation at AI speed: challenges and solutions}
    \label{tab:democratic-regulation}
\end{table*}
\section{Roadmap}
If FSA is meant to face these challenges, we must bring this work from basic research, through various legitimation steps, to policy proposals with broad support among experts and the public, and finally to broad implementation in new algorithms, institutions, and mechanisms—all in time to head off the socio-technical threats before AI agents cause widespread harms.

Theoretical research alone in the five areas outlined above is not sufficient. For an AI lab to adopt a mechanism built on these new foundations, three key conditions must be met:

\begin{enumerate}
\item Expert consensus must exist that the solution is the best available approach
\item A clear flagship implementation must exist as a concrete example
\item There must be widespread demand for the solution demonstrated by the flagship example
\end{enumerate}

The absence of any one condition significantly reduces the likelihood of adoption. Expert consensus without implementations risks being dismissed as theoretically intractable. Flagship examples without public demand will likely be deemed too risky for adoption. Public demand without flagship examples often fails to produce change.

The first phase focuses on developing rigorous theories, formalisms, and prototypes demonstrating the viability of explicit norm- and value-based AI and institution design. This requires validating prototypes through various legitimation tests—making formal claims about optimality and robustness while gathering data about small-scale deployments, user experience, and legitimacy.

In parallel, we must cultivate a strong interdisciplinary community—spanning scholars, technologists, and policymakers—to establish clear expert agreement around these novel frameworks, while launching targeted real-world pilots.This involves creating advisory networks and facilitating frequent meetings between researchers and implementers. Each solution will be distilled into concise policy or product recommendations supported by the relevant research.

We'll also need to find suitable venues for real-world deployments across our key domains:

\begin{itemize}
\item \textbf{Market interventions}: We can work with exchanges or market infrastructure providers to implement value-aligned mechanisms in real-world settings, perhaps as pilot projects or within specific submarkets.

\item \textbf{Democratic interventions}: We can engage with forward-looking jurisdictions (e.g., Estonia, Norway, Taiwan) prepared to pilot innovative approaches to AI governance, such as value-based AI regulatory infrastructure.

\item \textbf{Morally-competent agents}: We can partner with financial marketplaces or corporate suppliers willing to trial systems restricted to normatively competent AI agents.
\end{itemize}

These flagship implementations need not be deployed by the largest countries or top AI labs, but they must deliver robust successes that can generate positive feedback cycles between implementation success and broader adoption: successful implementations generate increased stakeholder demand, which in turn creates opportunities for more ambitious implementations across additional domains.
\section{Conclusion}
This paper has presented Full-Stack Alignment (FSA) as a framework for addressing the socio-technical challenge of co-aligning AI systems and institutions. We have argued that two dominant paradigms—preference/utility maximization inherited from the Standard Institution Design Toolkit (SIDT) and prompt- or self-critique-based text alignment—are structurally inadequate for robust socio-technical alignment. In their place, we have proposed explicit, structured representations of human norms and values that can be inspected, verified, and deliberated over.

\subsection{Summary of Contributions}

Our approach makes three primary contributions:

First, we have identified fundamental limitations in existing alignment paradigms. The preference-based approach fails to account for the instability of human preferences, their vulnerability to manipulation, and their inadequacy for modeling sophisticated cooperation. The text-based approach suffers from specification fragility, verification difficulties, and vulnerability to ideological pollution. Together, these limitations render current approaches insufficient for high-stakes alignment.

Second, we have developed a new toolkit centered on explicit norm and value representations. We outlined two fundamental approaches: (1) normative convergence that identifies attractors in value space which help distinguish genuine values from mere preferences, such as Kantian equilibria and self-other generalizations, and (2) structural representation that establishes formal encodings of values and norms, and methods for reasoning about them, as exemplified by constitutive attentional policies. These approaches enable us to distinguish genuine values from instrumental considerations, strategic concerns, and ideological markers.

Third, we demonstrated through five examples studies how these structured representations can address previously intractable alignment challenges: preventing value collapse in personal AI assistants, developing normatively competent agents, enabling win-win AI negotiation, creating a meaning-preserving economy, and designing democratic institutions that can operate at AI speed.

\subsection{Toward a New Institutional Paradigm}

Full-Stack Alignment is not merely a technical project but an institutional one. It calls for a reconfiguration of the relationship between AI systems and human institutions—a reconfiguration that preserves and enhances human agency rather than diminishing it. By moving beyond the limitations of preference satisfaction and text-based alignment, FSA opens the possibility of AI systems that genuinely serve human flourishing.

The path forward will require close collaboration among technical researchers, social scientists, ethicists, policymakers, and the broader public. It will demand both theoretical advances and practical experimentation in real-world settings. But the potential reward is significant: a technological future where AI systems and human institutions co-evolve in ways that strengthen rather than undermine our collective capacity to realize our deepest values.

If successful, this approach may contribute not only to AI alignment but to an institutional renewal that addresses longstanding limitations in how we collectively organize to pursue human flourishing. The explicit, accountable representation of norms and values offers a foundation not just for aligning AI, but for reimagining human institutions in an age of unprecedented technological change.
\section{Footnotes}
\begin{thebibliography}{99}

\bibitem{manin1997} Manin, B. (1997). \textit{The principles of representative government}. Cambridge University Press.

\bibitem{social-choice} Social choice started as non rooted in a rational model of humans: the first theorems (Condorcet) modeled humans as realization of some random variables and did not come with micro-foundations for these random processes. The 1950's social choice theory became more micro-founded/rational.

\bibitem{values-prefs} Or, more precisely, ideas like values, beliefs, and norms have just been folded into preferences. I don't mean to minimize the amount of work done on these topics: Outside the SIDT, these intersubjective concepts have been modeled in fields like sociology. And there is hugely successful work extending economics or game theory in these directions, but it remains peripheral. For instance, in microeconomics and game theory there are well-known theories of norm emergence; there are alternatives to rational choice theory like Ruth Chang's or Isaac Levi's. There are also larger efforts, like institutional and organizational economics. But (with the exception of transaction cost econ) these never make it into the SIDT that is used to design or justify institutions. Other innovations which do make it in (like behavioral economics or social network analysis) have remained context-free.

\bibitem{revealed-pref} When social theories equate revealed preference with benefit, anyone who can manipulate another's choice (which is supposed to reveal a certain preference) can count as benefiting them.

\bibitem{klingefjord} See ``Klingefjord, Lowe, and Edelman''

\bibitem{xuan} See Xuan...

\bibitem{sen} Amartya Sen

\bibitem{people-prefs} \textit{People's preferences change productively over time as they better understand what's important to them, face new contexts, and realize aspects of situations they didn't previously consider. That means that some people have better preferences in some contexts. But mechanisms based on these prevailing social theories can't detect them. These mechanisms miss the opportunity to harness moral learning, to let us reach collectively towards something that's not just the average.}

\bibitem{conitzer} Conitzer moral mechanism

\bibitem{klingefjord2} See ``Klingefjord, Lowe, and Edelman'' again

\bibitem{weyl} Glen Weyl put it well:
``[Currently models feature] a binary between Individuals, conceptualized as largely pre-social, independent ultimate loci of value/preference/good/belief, and some global coordination device variously referred to as the social planner, the mechanism designer, the impartial observer, God or, most commonly, The State.

It is from this binary that the initially paradoxical combination of extreme individualism and authoritarian technocracy that characterizes [this view] arises.''

\bibitem{edelman} A longer treatment of these ideas is in ``Nothing to Be Done'' by Joe Edelman. Look for the paragraph that starts:
``Each of these fields is subject to another kind of test, besides its ability to predict human choices. Each field has a model of human beings which focuses on certain skills or abilities or activities of the human subject.''

\bibitem{terms} Some of these terms allow people to advocate for a policy direction, some allow them to find each other, some allow them to advocate for leaders that serve their previously-unnameable interests. `Modern Social Imaginaries /`

\bibitem{voting} There's even a tendency to say something like ``I guess if people are voting for social isolation, with their dollars and at the ballot, then a good society is one of social isolation! (So long as it's free and fair.)''

\bibitem{taylor} [Taylor, ``Modern Social Imaginaries''] Taylor argues that historically, this co-option ends up undermining the power of the incumbent system.

\bibitem{effects} These effects are considered fairly well-supported in political theory and political science. [Weber, Taylor, some recent political scientists]

\end{thebibliography}

\begin{small}
\textbf{Rhetoric / Use Flywheel:}

Let us suppose this pans out. This will then change people's ideas of what a good society looks like. Currently, the dominant social ideals are freedom and fairness. It is not so surprising that—in a society dominated by markets and voting, where we think of ourselves as entrepreneurs and citizens—our ideal society is one of freedom and fairness. Other notions—like having meaningful lives, advancing science and art, having strong communities—these seem attractive, but they don't fit as cleanly into our markets/voting-dominated vocabulary as freedom and fairness do.

But note that the this depends on the institutional mix. Other mechanisms lead to other ideals. For instance, the spread of wikis and open source has popularized the idea that a good society should be open to 'pull requests' of some form, leading to 'read/write society' wiki-based governance ideas in Taiwan and other places. One can imagine that Twitter's Community Notes might popularize a notion that a good society is 'algorithmically sober'.

In general, whenever a new institution seems better along a \textit{new dimension} (it is not just more free or more fair), a new rhetoric emerges. It's unlikely that 'read/write society' or 'algorithmically sober' will overtake freedom and fairness as the qualities most heavily advocated for in social arrangements, but we think that some other term, not coined yet, probably will. This is how political paradigm shifts have always gone.

We don't know which new terms will take hold—they might be about ``surfacing collective wisdom,'' ``strengthening social fabric,'' or ``enabling diverse forms of life''—but they will become frameworks for reimagining social possibilities.

Should such a positive feeling and new rhetoric emerge around our flagship deployments, we will encourage it. We will help encourage participants to articulate the advantages of these new systems, and share what it's like to participate in them.

Terminology can be a decisive factor in which groups manage to coordinate politically. For instance, it's hard for the working class to find each other without the term 'working class'; it's hard to fight for collective ownership without ideas like 'worker coop'; it's hard to fight for local self-determination without concepts like 'freedom', 'subsidiarity', or 'laissez-faire'.

Many communities would ideally band together in common interest, but for now they can't even find one another, or have nothing clear to advocate for. There are no terms yet.

Once there are new terms—once people begin to see themselves as a ``values-driven'' (rather than just as entrepreneurs, informed citizens, Democrats, or Republicans) and band together with other values-driven people to further values-driven forms of social organization—it will be hard to reverse.

Eventually, the legitimacy of political arrangements will be defined by a vocabulary which today isn't even in play.
\end{small}
\section{Related Work}
The literature on alignment, mechanism design, ethics, and multi‑agent learning is vast. This appendix situates \textit{Full‑Stack Alignment (FSA)} concepts alongside the closest ideas in adjacent fields and highlights the delta each contribution adds. The table is meant as a navigational aide rather than an exhaustive survey.

\begin{table*}[t]
\centering
\small
\begin{tabular}{p{0.22\textwidth}p{0.22\textwidth}p{0.22\textwidth}p{0.22\textwidth}}
\toprule
\textbf{FSA concept} & \textbf{Closest idea in economics / mechanism design} & \textbf{Closest idea in AI / cooperative AI} & \textbf{Key delta introduced by FSA} \\
\midrule
\textbf{Constitutive attentional policy} (value \textit{card}) & Preference refinement / lexicographic utilities & RLHF preference model; Chain‑of‑Thought rationales & Explicitly distinguishes \textit{constitutive} from \textit{instrumental} considerations and is designed for auditability \& endorsement, not just optimisation. \\
\midrule
\textbf{Moral graph} (value graph at population scale) & Welfare function with interpersonal comparisons; Social welfare ordering & Value aggregation via voting models, e.g. quadratic voting; democratic RL & Captures \textit{relationships} (refines, conflicts, supersedes) among values + supports contestability; not a single scalar objective. \\
\midrule
\textbf{Attractors in value space / self‑other generalisation} & Kantian equilibrium (Roemer 2010), dependency equilibria & Cooperative MARL regularisers (e.g. LOLA, Opponent Shaping) & Treats convergence criteria as \textit{normative} fixed points rather than merely strategic equilibria. \\
\midrule
\textbf{Integrity \& value revelation} & Costly signalling; cheap‑talk equilibria with credence & Open‑source game theory; model watermarking & Middle ground: share structured value commitments (partial source code) that are verifiable but keep strategic internals private. \\
\midrule
\textbf{Norm‑augmented Markov games} & Social \& moral preferences in repeated games & Normative RL, Social Influence bonus & Formal separation between individual pay‑offs and population norms enables rapid norm learning \& adaptation. \\
\midrule
\textbf{Resource‑rational contractualism} & Rawlsian contractarianism, bargaining solutions & Virtual bargaining (Jara‑Ettinger et al.) & Implements \textit{bounded} universalisation that scales to LLM reasoning budgets. \\
\midrule
\textbf{Outcome‑based contracting for meaning} & Dynamic principal–agent contracts & Alignment via reward models of human satisfaction & Pay‑outs keyed to \textit{measured flourishing} using structured value schema; resists preference manipulation. \\
\midrule
\textbf{Democratic regulator at AI speed} & Delegative / liquid democracy; participatory budgeting & Alignment assemblies; LLM policy bots & Builds legitimacy on population‑level moral graph and provides verifiable reasoning traces in real time. \\
\midrule
\textbf{Values‑interpretable architectures} & Mechanism transparency (auditability) & Interpretable self‑explaining neural networks & Ties explanations to \textit{formal value objects} rather than post‑hoc saliency maps. \\
\midrule
\textbf{Thick evaluative concept competence} & Sen's capability approach & Expression of moral concepts in LLMs (Anthropic, DeepMind studies) & Requires models to map concepts to structured attentional policies, enabling verification and refinement. \\
\bottomrule
\end{tabular}
\caption{Comparison of FSA concepts with related work in economics and AI}
\label{tab:related_work}
\end{table*}

\textit{Citations (representative):} Gul \& Pesendorfer 2001; Bewley 2002; Fehr \& Schmidt 1999; Tversky \& Kahneman 1992; Roemer 2010; Oesterheld 2022; Jara‑Ettinger 2023; Goel et al. 2019; Anthropic 2022.

\renewcommand{\bibsection}{\section*{References}}
\setlength{\bibsep}{0.0pt}
\bibliographystyle{plainnat}
\bibliography{references}

\end{document}